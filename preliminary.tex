
%\section {Preliminaries}\label{sec:pre}

\section{Preliminaries}\label{sec:pre}
Given a set $\mathcal{P}$ of atomic propositions, an \ltlf formula
$\phi$ has the form:

%$\phi ::= \true\ |\ \ff\ |\ p\ |\ \neg \phi\ |\ \phi\vee\phi\ |\ \phi\wedge\phi\|\ X\phi\ |\ N\phi\ |\ \phi U\phi\ |\ \phi R\phi$,
$\ \ \ \ \ \ \ \ \ \ \ \phi ::= \true\ |\ p\ |\ \neg \phi\ |\ \phi\wedge\phi\ |\ \X\phi\ |\ \phi \U\phi$;\\
%
where $\true$ is true, $\neg$ is the negation operator, $\wedge$ is the and operator, $\X$ is the strong Next operator 
and  $\U$ is the Until operator. We also have the duals $\ff$ (false) for $\true$, $\vee$ for $\wedge$, 
$\N$ (weak Next) for $\X$ and $\R$ for $\U$. A \emph{literal} is an atom $p\in\mathcal{P}$ or its negation ($\neg p$).
 Moreover, we use the notation $\G\phi$ (Globally) and $\F\phi$
(Eventually) to represent $\ff \R\phi$ and $\true \U\phi$.
%We have $N\phi \equiv \neg X\neg \phi$ and
%$\phi_1 R\phi_2\equiv \neg(\neg\phi_1 U\neg\phi_2)$. Note that in %$\ltlf$, $X\phi\equiv N\phi$ is not true, which is however the case %in LTL. 
Notably, $\X$ is the standard \emph{next} operator, while $\N$ is \emph{weak next}; 
$\X$ requires the existence of a successor state, while $\N$ does not. 
Thus $\N\phi$ is always true in the last state of a finite trace, since no successor exists there.
This distinction is specific to \ltlf.

\ltlf formulas are interpreted over finite traces \cite{GV13}.
Given an atom set $\mathcal{P}$, we define $\Sigma = 2^{\mathcal{P}}$ be  the family of sets of atoms. Let
$\eta\in\Sigma^+$ be a finite nonempty trace, with $\eta=\sigma_0\sigma_1\ldots\sigma_n$. we use
$|\eta|=n+1$ to denote the length of $\eta$. Moreover, for $0\leq
i\leq n$, we denote $\eta[i]$ as the i-th position of $\eta$, and 
$\eta_i$ to represent $\sigma_i\sigma_{i+1}\ldots\sigma_n$, which is
the suffix of $\eta$ from position~$i$. We define the satisfaction relation $\eta\models\phi$ as follows:

\begin{itemize}%[noitemsep,topsep=0pt]
  \item $\eta\models\true$; and $\eta\models p$, if $p\in\mathcal{P}$ and $p\in\eta[0]$;
  \item $\eta\models\neg\phi$, if $\eta\not\models\phi$;
  \item $\eta\models\phi_1\wedge\phi_2$,  if $\eta\models\phi_1$ and $\eta\models\phi_2$;
  %\item If $\phi=\phi_1\vee\phi_2$, $\eta\models\phi$ if $\eta\models\phi_1$ or $\eta\models\phi_2$.
  \item $\eta\models \X\phi$ if $|\eta|>1$ and $\eta_1\models\psi$;
  %\item If $\phi=N\psi$, $\eta\models\phi$ if $|\eta|>1$ implies $\eta_1\models\psi$;
  \item $\eta\models (\phi_1 \U\phi_2)$, if there exists $0\leq i < |\eta|$
  such that $\eta_i\models\phi_2$ and for every $0\leq j < i$ it holds that $\eta_j\models\phi_1$;
 % \item If $\phi=(\phi_1 R\phi_2)$, $\eta\models\phi$ if for every $0\leq i <|\eta|$ either $\eta_i\models\phi_2$ or for
  %some $0\leq j < i$ it holds that $\eta_j\models\phi_1$.
  \end{itemize}


\begin{definition}[\ltlf Synthesis]\label{def:synthesis}
Let $\phi$ be an \ltlf formula with the atomic set $\P$ and $\X,\Y$ be two disjoint atomic sets such that $\X\cup \Y= \P$. $\X$ is the set of input variables and $\Y$ is the set of output variables. $\phi$ is realizable with respect to $\langle \X,\Y\rangle$ if 
\begin{itemize}
\item for the \textbf{Environment-first} synthesis, there exists a strategy $g: (2^\X)^* \to 2^\Y$, such that for an arbitrary infinite sequence $\lambda = X_0, X_1, \cdots \in (2^\X)^\omega$ of propositional interpretations over $\X$, we can find $k \geq 0$ such that $\rho\models\phi$ is true, where $\rho=(X_0\cup g(\epsilon)),(X_1\cup g(X_0)),\cdots,(X_k\cup g(X_0,\cdots,X_{k-1}))$. ($\epsilon$ means the empty trace.) 
\item for the \textbf{System-first} synthesis, there exists a strategy $g: (2^\X)^+ \to 2^\Y$, such that for an arbitrary infinite sequence $\lambda = X_0, X_1, \cdots \in (2^\X)^\omega$ of propositional interpretations over $\X$, we can find $k \geq 1$ such that $\rho\models\phi$ is true, where $\rho=(X_0\cup g(X_0)),(X_1\cup g(X_1)),\cdots,(X_k\cup g(X_0,\cdots,X_{k}))$. 
\end{itemize}
\end{definition}

In this paper, we focus on the \textbf{Environment-frist} \ltlf synthesis. Also, we fix the notations $\X$ and $\Y$ for the variables controlled by the environment and system respectively in the \ltlf synthesis.

\noindent\textbf{Notations.} 
We say an \ltlf formula is in the \emph{Negated Normal Form} (\NNF), if the negation operators appear only in front of an atom. It is trivial to know that every \ltlf formula can be converted into its \NNF at a linear cost. 
We use $cl(\phi)$ to denote the set of subformulas of $\phi$. 
Let $A$ be a set of \ltlf formulas, we denote $\bigwedge A$ to be the formula $\bigwedge_{\psi\in A}\psi$.
The two \ltlf formulas $\phi_1,\phi_2$ are semantically equivalent, denoted as $\phi_1\equiv\phi_2 $, iff for every finite trace $\eta$, $\eta\models\phi_1$ iff $\eta\models\phi_2$.  Obviously, we have $(\phi_1\vee\phi_2)\equiv \neg (\neg\phi_1 \wedge\neg \phi_2)$, 
$\N\psi\equiv \neg \X\neg\psi$ and $(\phi_1 \R\phi_2)\equiv \neg (\neg\phi_1 \U\neg \phi_2)$.




