\section{Appendix}
\subsection{Proof of Theorem \ref{thm:ltlf2tdfa}}

We first introduce the necessary preliminaries to prove this theorem. 

To leverage \SAT solvers for the automata construction, we first need to consider \ltlf formulas as propositional formulas. The motivation comes from treating all temporal subformulas of $\phi$ as atomic propositions. 

\begin{definition}[Propositional Atoms]\label{def:pa}
For an \ltlf formula $\phi$, we define the set of propositional atoms, $\PA(\phi)$, recursively as follows.
\begin{itemize}
	\item If $\phi$ is an atom, Next, weak Next, Until or Release formula, $\PA(\phi)=\{\phi\}$;
	\item If $\phi = (\neg \psi)$, $\PA(\phi)= \PA(\psi)$;
	\item If $\phi = (\phi_1 \wedge \phi_2)$ or $(\phi_1 \vee \phi_2)$, $\PA(\phi) = \PA(\phi_1) \cup \PA(\phi_2)$.
\end{itemize}
\end{definition}

Consider the formula $\phi$ = $(a \wedge ((\neg c \wedge a)\U b) \wedge (c \wedge \X (a \vee b)))$ as an example. According to Definition \ref{def:pa}, we have $\PA(\phi)=\{a, c, (\neg c \wedge a)\U b), (\X (a \vee b))\}$. Now we define the propositional formulas over propositional atoms w.r.t. an \ltlf formula.

\begin{definition}[Propositional Formulas]\label{def:pf}
Given an \ltlf formula $\phi$, let $\phi^{p}$ be a propositional formula over $\PA(\phi)$. 
\end{definition}

Consider $\phi$ = $(a \wedge ((\neg c \wedge a)\U b) \wedge (c \wedge \X (a \vee b)))$. From Definition \ref{def:pf}, the corresponding propositional formula $\phi^p = (a \wedge p_1) \wedge (c \wedge p_2)$, in which $p_1, p_2$ are two Boolean variables representing the truth values of $((\neg c \wedge a)\U b)$ and $(c \wedge \X (a \vee b))$. We define the \emph{next normal form} for an \ltlf formula, whose corresponding propositional formula is the actual input to \SAT solvers.

\begin{definition}[neXt Normal Form]\label{def:xnf}
For an \ltlf formula $\phi$, we define its Next Normal Form, i.e., $\xnf(\phi)$, recursively as follows. 
\begin{itemize}
\item $\xnf{\tt} = \tt$ and $\xnf{\ff} = \ff$;
\item If $\phi$ is a literal, $\xnf{\phi} = \phi \wedge \X(\tt)$;
\item If $\phi = \X(\psi)$ or $\phi = \N(\psi)$, $\xnf{\phi} = \X(\psi)$;
\item If $\phi = (\phi_1 \wedge \phi_2)$, $\xnf{\phi} = \xnf{\phi_1}\wedge\xnf{\phi_2}$;
\item If $\phi = (\phi_1 \vee \phi_2), \xnf{\phi} = \xnf{\phi_1}\vee \xnf{\phi_2}$; 
\item If $\phi = (\phi_1 \U \phi_2), \xnf{\phi} = \xnf{\phi_2} \vee (\xnf{\phi_1} \wedge \X{\phi})$; 
\item If $\phi = (\phi_1 \R \phi_2), \xnf{\phi} = \xnf{\phi_2}\wedge (\xnf{\phi_1} \vee \X{\phi})$.
\end{itemize}
\end{definition}

By restricting the \ltlf formula $\phi$ to the \XNF, we can obtain a satisfying assignment of $\phi^{p}$ with the help of the SAT solver.
For a satisfying assignment $A$, we define $X(A)=\{\theta\ |\ \X\theta\in A\}$ and $L(A) =\{l\ |\ l \in A \textit{ is a literal}\}$. Also for the sake of simplicity, \textbf{we mix the usage of $X(A)$ to denote the conjunction of all elements in the set, i.e., $\bigwedge_{\psi_i\in X(A)}\psi_i$. The same applies to $L(A)$.}
The following lemma describes the relationship between an \ltlf formula and the satisfying assignment of its corresponding propositional formula.

\begin{lemma}\label{lem:propSat}
Given an \ltlf formula $\phi$ in the \XNF and a finite trace $\eta\in\Sigma^+$ with $|\eta|>1$, $\eta\models\phi$ holds iff there is a satisfying assignment $A$ of $\phi$ such that $\eta[0]\models L(A)$ and $\eta_1\models X(A)$.
\end{lemma}
\begin{proof}
$(\Rightarrow)$ The proof can be achieved by induction over the types of $\phi$. Since $\phi$ is in \XNF, $\phi$ cannot be an Until or Release formula.
\begin{itemize}
	\item if $\phi$ is a literal,  we could have a satisfying assignment $A=\{\phi\}$ for $\phi$. In this situation, $L(A)=\{\phi\}$ and $X(A)=\{\tt\}$. Therefore, it is true that $\eta[0]\models L(A)$ and $\eta_1\models X(A)$;
	\item if $\phi=\X\psi$, we could have a satisfying assignment $A=\{\X\psi\}$ for $\phi$. In this situation, $L(A)=\{\tt\}$ and $X(A)=\{\psi\}$. Considering we have $\eta\models\phi$ i.e., $\eta\models\X\psi$, it is true that $\eta[0]\models L(A)$ and $\eta_1\models X(A)$;
	\item if $\phi = \phi_1\wedge\phi_2$, $\eta\models\phi$ 
implies $\eta\models\phi_1$ and $\eta\models\phi_2$. By assumption hypothesis, there is $A_i$ of $\phi_i$ ($i=1,2$) such that 
$\eta[0]\models L(A_i)$ and $\eta_1\models X(A_i)$. Let $A = A_1\cup A_2$, and a consistent $A$, in which either $\psi$ or $\neg \psi$ cannot be together, 
must exists ($A$ may not be unique because $A_1$ and $A_2$ may not be unique). 
Otherwise, there is $\psi\in A_1$ and $\neg\psi\in A_2$ 
such that $\eta$ cannot model $\bigwedge A_1$ and $\bigwedge A_2$ at the same time, which is a contradiction. So $A$ is a satisfying  assignment of $\phi$, which leads to $\eta[0]\models L(A)$ and $\eta_1\models X(A)$. 
\item The proof for $\phi=\phi_1\vee\phi_2$ is analogous to the proof for $\phi = \phi_1\wedge\phi_2$.
\end{itemize} 

$(\Leftarrow)$ Let $A$ be a satisfying assignment of $\phi$, so $A\models\phi$ implies $(\bigwedge A)\Rightarrow \phi$. 
Also, $\eta[0]\models L(A)$ and $\eta_1\models X(A)$ implies $\eta\models \bigwedge A$. Therefore, it is true that $\eta\models \phi$.
\end{proof}

Lemma \ref{lem:propSat} does not cover the case when $|\eta|=1$, which yields the following corollary. 

\begin{corollary}\label{coro:propSat}
Given an \ltlf formula $\phi$ in the \XNF and a finite trace $\eta\in\Sigma^+$ with $|\eta|=1$, $\eta\models\phi$ holds implies there is a satisfying assignment $A$ of $\phi$ such that $\eta[0]\models L(A)$.
\end{corollary}
\begin{proof}
The proof is similar to that of Lemma \ref{lem:propSat} for the ($\Rightarrow$) direction, the details of which are omitted here.
\end{proof}

Now we start to prove Theorem \ref{thm:ltlf2tdfa}
\iffalse
\begin{theorem}
Given an \ltlf formula $\phi$ and the \TDFA ${\A_{\phi}}$ constructed by Definition \ref{def:ltlf2dfa}, a finite trace $\eta\models\phi$ holds iff $\eta$ is accepted by ${\A_{\phi}}$. 
\end{theorem}
\fi
\begin{proof}
Assume $\eta = \omega_0\omega_1\ldots\omega_n$ ($n\geq 0$). 

($\Leftarrow$) 
According to Definition \ref{def:ltlf2dfa},  $\eta$ accepted by ${\A_{\phi}}$ implies that there is a run $r=s_0s_1\ldots s_n s_{n+1}$ of ${\A_{\phi}}$ on $\eta$ such that $s_{n}\tran{\omega_{n}}s_{n+1} \in \delta$ and $\omega_n\models s_n$. To prove $\eta$ is accepted by ${\A_{\phi}}$ implies $\eta\models\phi$ holds, we only need to prove $\eta$ is accepted by ${\A_{\phi}}$ implies $\eta_{n-i}\models s_{n-i}$ for\ i=0...n. Then we could have $\eta (=\eta_0)\models \phi (=s_0)$ when $i=n$.

We prove by induction over the values of $i$.
\begin{itemize}
	\item By Definition \ref{def:ltlf2dfa}, we have that $\omega_n\models s_n$, which means basically when $i = 0$, it holds that $\eta_{n-0}\models s_{n-0}$;
	\item Inductively, assume $\eta_{n-i}\models s_{n-i}$ holds($0<i\leq n$). Because $s_{n-i-1}\tran{\omega_{n-i-1}}s_{n-i}$ is a transition in the \TDFA and $\eta_{n-i}\models s_{n-i}$ holds, the set $A_{n-i-1} = \omega_{n-i-1}\cup \{\X\theta | \theta\in s\}$ is a satisfying assignment of $\xnf{s_{n-i-1}}$ in $\SAT(\xnf{s_{n-i-1}})$, from Definition \ref{def:ltlf2dfa}. Then based on Lemma \ref{lem:propSat} and since $\eta_{n-i-1}=\omega_{n-i-1}\cdot\eta_{n-i}$ is true, we have that $\eta_{n-i-1}\models s_{n-i-1}$.
\end{itemize}

($\Rightarrow$) We first prove that $\eta\models\phi$ implies that there is a run $r$ of ${\A_{\phi}}$ on $\eta$. 
If $n = 0$, Corollary \ref{coro:propSat} shows that there is a satisfying assignment $A$ of $\xnf{\phi}$ such that $\eta[0]\models L(A)$. From Definition \ref{def:ltlf2dfa}, $\phi\tran{\eta[0]}s$, where $s=\{X(A') | A'\in\SAT(\xnf{\phi})\}$, is a transition of the \TDFA. As a result, there is a run $r=s_0(=\phi)s$ of ${\A_{\phi}}$ on $\eta$.
 
If $n>0$, according to Lemma \ref{lem:propSat}, $\eta\models\phi$ implies there is a satisfying assignment $A$ of $\xnf{\phi}$ such that $\omega_0\models L(A)$ and $\eta_1\models X(A)$. From Definition \ref{def:ltlf2dfa}, $\phi (=s_0)\tran{\omega_0}s_1$, where $s_1=\{X(A') | A'\in\SAT(\xnf{\phi})\}$, is a transition in the \TDFA. Recursively applying the above process to $\eta_i\models s_i(i\geq 1)$, one can prove there is a transition $s_i\tran{\omega_i}s_{i+1}$ in the \TDFA. As a result, we have $\eta\models\phi$, which implies there is a run $r$ of ${\A_{\phi}}$ on $\eta$.  

Now we have that $\eta\models\phi$ implies there is a run $r=s_0...s_n s_{n+1}$ of ${\A_{\phi}}$ on $\eta$ and $\eta_i\models s_i\ for\ i=0...n$. When $i=n$, $\eta_n\models s_n$ i.e., $\omega_n\models s_n$ holds. According to Definition \ref{def:ltlf2dfa}, the transition $s_n\tran{\omega_n}s_{n+1}$ is an accepting transition. As a result, $r$ is an accepting run and $\eta$ is accepted by ${\A_{\phi}}$.
\end{proof}
