\section{Introduction}\label{sec:intro}

Linear Temporal Logic over Finite Traces, or $\ltlf$, is a formal language gaining popularity in the AI community for formalizing and validating system behaviors. While standard Linear Temporal Logic (LTL) is interpreted on infinite traces \cite{Pnu77}, $\ltlf$ is interpreted over finite traces \cite{GV13}.  While LTL is typically used in formal-verification settings, where we are interested in nonterminating computations, cf. \cite{Var07a},  $\ltlf$ is more attractive in AI scenarios focusing on finite behaviors, such as planning \cite{BK98,DV99,CDV02,PLGG11,CBMM17}, plan constraints \cite{BK00,Gab04}, and user preferences \cite{BFM06,BFM11,SBM11}. Due to the wide spectrum of applications of $\ltlf$ in the AI community \cite{DMM14}, it is worthwhile to study and develop an efficient framework for solving $\ltlf$-reasoning problems. Just as propositional satisfiability checking is one of the most fundamental propositional reasoning tasks, $\ltlf$ satisfiability checking is a fundamental task for $\ltlf$ reasoning. 

Given an $\ltlf$ formula, the satisfiability problem asks whether there is a finite trace that satisfies the formula. A ``classical'' solution to this problem is to reduce it to the LTL satisfiability problem \cite{GV13}. The advantage of this approach is that the LTL satisfiability problem has been studied for at least a decade, and many mature tools are available, cf.~\cite{RV07,RV12}. Thus, $\ltlf$ satisfiability checking can benefit from progress in LTL satisfiability checking. There is, however, an inherent drawback that an extra cost has to be paid when checking LTL formulas, as the tool searches for a ``lasso'' (a lasso consists of a finite path plus a cycle, representing an infinite trace), whereas models of $\ltlf$ formulas are just finite traces. Based on this motivation, \cite{LZPVH14} presented a tableau-style algorithm for $\ltlf$ satisfiability checking. They showed that the dedicated tool, \emph{Aalta-finite}, which conducts an explicit-state search for a satisfying trace, outperforms extant tools for $\ltlf$ satisfiability checking. 
%In particular, \emph{Aalta} outperforms LTL satisfiability checking performed via a reduction to BDD-based symbolic model checking \cite{RV11}, which was until then the best approach to LTL satisfiability checking \cite{RV11}.


The conclusion of a dedicated solver being superior to $\ltlf$ satisfiability checking from~\cite{LZPVH14}, seems to be out of date by now because of the recent dramatic improvement in propositional SAT solving, cf.~\cite{MZ09}. On one hand, SAT-based techniques have led to a significant improvement on LTL satisfiability checking, outperforming the tableau-based techniques of \emph{Aalta-finite} \cite{LZPVH14}. (Also, the SAT-based tool \emph{ltl2sat} for $\ltlf$ satisfiability checking outperforms \emph{Aalta-finite} on particular benchmarks \cite{FG16}.) On the other hand, SAT-based techniques are now dominant in symbolic model checking \cite{CCDGMMMRT14,VWM15}. Our preliminary evaluation indicates that $\ltlf$ satisfiability checking via SAT-based model checking  \cite{Bra11,EMB11} or via SAT-based LTL satisfiability checking \cite{LZPV15} both outperform the tableau-based tool \emph{Aalta-finite}.
Thus, the question raised initially in \cite{RV07} needs to be re-opened with respect to $\ltlf$ satisfiability checking: is it best to reduce to SAT-based model checking or develop a dedicated SAT-based tool?

Inspired by \cite{LZPV15}, we present an explicit-state SAT-based framework for $\ltlf$ satisfiability. We construct the \emph{$\ltlf$ transition system} by utilizing SAT solvers to compute the states explicitly. Furthermore, by making use of both satisfiability and unsatisfiability information from SAT solvers, we propose a \emph{conflict-driven} algorithm, \cdlsc, for efficient $\ltlf$ satisfiability checking. We show that by specializing the transition-system approach of \cite{LZPV15} to $\ltlf$ and its finite-trace semantics, we get a framework that is significantly simpler and yields a much more efficient algorithm \cdlsc than the one in \cite{LZPV15}.

We conduct a comprehensive comparison among different approaches. 
Our experimental results show that the performance of \cdlsc dominates all other existing $\ltlf$-satisfiability-checking algorithms. On average, \cdlsc achieves an approximate four-fold speed-up, compared to the second-best solution (IC3 \cite{Bra11}+K-LIVE \cite{CS12}) tested in our experiments. Our results re-affirm the conclusion of \cite{LZPVH14} that the best approach to $\ltlf$ satisfiability solving is via a dedicated tool, based on explicit-state techniques.

%In the rest of this paper, missing proofs are shown in the supplemental material. 

%The rest of this paper is organized as follows. Section Preliminaries  introduces definitions for $\ltlf$ and its satisfiability problem. Then we describes SAT-based algorithms for $\ltlf$ satisfiability checking. Finally we domenstrates the experimental results and concludes the paper. Due to the page limit, all proofs are omitted.
