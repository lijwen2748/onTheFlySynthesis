\section{Introduction}\label{sec:intro}
Synthesis for Linear Temporal Logic over finite traces, e.g., \ltlf\citep{GV13}, has emerged as a popular research topic in AI community due to its applications such as motion planning \citep{ZGPV20,R04}.  
\ltlf is a formal logic that receives lots of concerns from the AI community, for its simplicity to formalize and validate behaviors of AI systems. While standard Linear Temporal Logic (\ltl) is interpreted on infinite traces \citep{Pnu77}, \ltlf is interpreted over finite traces \citep{GV13}. Since first introduced in 2013, the fundamental problems of \ltlf, e.g., satisfiability\citep{LRPZV19} and realizability\citep{Vardi95,GV15,ZTLPV17,PR89}, are extensively studied in previous literature. Towards applications, researchers successfully reduce the planning problem with \ltlf specifications to the synthesis for \ltlf realizability \citep{CTMBM17,ZGPV20,CBM18,AGMR18,AGMR19}, which makes the logic extremely attractive in the domain. We focus on the synthesis problem for \ltlf in this paper.   

Given an \ltlf formula $\phi$ with two atomic set $\X,\Y$ such that $\X\cap \Y=\emptyset$ and $\P=\X\cup\Y$ is the atomic set of $\phi$, the synthesis problem asks informally whether every finite trace generated as the result of the inter-actions between variables out of $\X$ and $\Y$, can satisfy the formula $\phi$ (see Definition \ref{def:synthesis} for details). The extant solution reduces \ltlf synthesis to the \dfa game\citep{BM06,Geffner13}. Explicitly, the \dfa that recognizes the same language as the \ltlf formula has to be constructed at first. Then a fix-point calculation is performed on the generated \dfa back from the accepting states of the automaton. The calculation initially marks the accepting states as the \emph{winning} states, and recursively extends such winning set based on the current information. If finally the initial state of the \dfa is included in the winning set, the conclusion reaches that the formula $\phi$ is realizable w.r.t. the atomic sets $\X$ and $\Y$. Otherwise, the result is unrealizable. It is well-known that the \ltlf-to-\dfa translation is the most consuming part, as the size of the generated \dfa can be at most doubly exponential to the size of the formula\citep{KV01d}. Indeed, the \dfa construction now becomes the main bottleneck of \ltlf synthesis.  

Several optimizations on the \dfa construction have been proposed. \citep{ZTLPV17} shows that generating minimal \dfa can be advantageous for \ltlf synthesis, based on the fact that using \mona tool \citep{HJJKPRS95,EKM98} to construct minimal explicit \dfa can be much faster than using \spot \citep{DP04} to construct symbolic but non-minimal ones. \citep{TV19} introduced the partition technique to decompose the generation of a large \dfa into small ones\citep{MSL18}. Recently, \citep{BLTV20} combines both the explicit and symbolic representations for \dfa state-space to successfully achieve a better \dfa construction than those using only one single representation. Nonetheless, none of the techniques above can avoid the double exponential-up cost, as the \dfa construction is the 
indispensable process for \ltlf synthesis. Therefore, the question raises up that, is it possible to produce \ltlf synthesis without generating the whole \dfa? 

In this paper, we present our solutions to achieve such goal. First of all, we present several light-weight pre-processing techniques for \ltlf synthesis, which can be easily implemented and with the low running-cost. Notably, such pre-processing techniques can be performed directly on the input formula even without constructing \dfa. As a result, they can be integrated into all other available \ltlf synthesis approaches. Secondly, we propose a new synthesis framework that is based on \dfa construction on the fly, i.e., states of the \dfa are created only necessary. In a high-level description, we start forward from the initial state $s_0$ (that is the input formula $\phi$ under our framework)  and continuously generate successor states when necessary. As soon as the new created state is determined as \emph{winning/failure}, a backtrack procedure is invoked to determine whether the predecessors are winning/failure based on the obtained information. As soon as the initial state can be determined as winning (resp. failure), the realizable (resp. unrealizable) result can be conclude. In the worst case, the algorithm returns unrealizable when the whole \dfa is constructed. 

We conducted an extensive experimental evaluations on both the pre-processing techniques and the new synthesis framework, by comparing to the extant \ltlf synthesis tools \syft\citep{ZTLPV17} and \lisasyft \citep{BLTV20}. The results show that: (1) the pre-processing techniques can speed-up the synthesis with up to an exponentially better performance; (2) even though cannot dominate the extant approaches, the new synthesis framework via on-the-fly \dfa construction can involve a significant improvement on the scalability performance.  